\chapter{Indledning}
\label{cha:indledning}
I dette afsnit vil der blive beskrevet hvad en enkoder er og hvilke typer enkodere der findes.\\[5mm]

Der er to forskellige slags enkoder-typer. En absolut enkoder og en inkrementel enkoder.\\[5mm]

Den absolutte enkoder gemmer sin position når strømmen fjernes fra systemet og det er muligt at få fat i denne position straks efter strømmen til systemet er etableret. Forholdet mellem den fysiske position og den værdi enkoderen har er noget der bliver lavet når enkoderen samles. En absolut enkoder behøves ikke at have et bestemt reference punkt, som er et fast punkt, som enkoderen bruger til at vide, hvor den befinder sig.\\[5mm] 

En inkrementel enkoder derimod har ikke et fast forhold mellem den fysiske position og enkoderen værdi men kan ændre sig fra den ene gang til den anden. Den inkrementelle enkoder kan have behov for at navigere til referencepunktet for at den nøjagtigt kan registrere positioner.\\[5mm]

Selve opbygningen af disse enkoder-typer kan enten være en mekanisk eller en optisk… når jeg finder ud af forskellen kommer der mere!!!

%%% Local Variables:
%%% mode: latex
%%% TeX-master: "../../master"
%%% End: